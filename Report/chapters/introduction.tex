\section{start of intro}

\section{Rules of Go}

The rules of Go are extremely simple, and their simplicity allows a real complexity and depth of play. The game is played on a so called go-ban, which are board composed of 9*9,13*13 or 19*19 intersections. There are two players, one who play the white stones, the other the black stones. Each player puts a stone on an intersection on the go-ban alternatively.They are two important concept to understand, the liberties and the group. 
\\

A stone has as many liberties as there are free intersections next to it. In figure 1, the black stone has initially four liberties, and it progressively decreases to one : 

\begin{figure}
\begin{center}
\includegraphics[scale=0.5]{Golibs}
\includegraphics[scale=0.5]{gogroupe}
\caption{Liberties of a stone and of a group}
\end{center}

\end{figure}
Two or more stones can fusion together to form a group in which they share the liberties. Here we can see three group, one black and two white. The black one has five liberties by example. 


An important concept is the capture, the board being representative(as a lot of ancient board games, like chess e.g) of a real situation of war. If a group is surrounded by the opponent, understand has no more liberty, it's dead. All the stones of the group are taken as prisoners, leaving free the intersections where they used to be. 

In order to win, a player must have more points than the opponent. A player gains one point by prisoner and one point by territory. A territory is a free intersection surrounded at least partially by your stones and potentially by the sides of the board. 
\\

In the example in \ref{score}, the white player win with 10 points ( 8 from territories and 2 from prisoners) against 	6 points for black( 5 from territories and 1 prisoner ). 
\begin{figure}
\begin{center}
\includegraphics[scale=0.5]{scorego.PNG}
\end{center}
\caption{Scoring}
\label{score}
\end{figure}
\section{start of intro}

\section{Rules of Go}

The rules of Go are extremely simple, and their simplicity allows a real complexity and depth of play. The game is played on go-ban, which are board of 9*9,13*13 or 19*19 intersections. There is two players, one who play the white stones, the other the black stones. Each player put a stone on an intersection on the go-ban alternatively.They are two important concept to understand, the liberties and the group. 
\\

A stone has as many liberties as free intersection next to it. In the following case, the black stone has initially four liberties, and it progressively decrease to one : 

\begin{center}
\includegraphics[scale=0.5]{Golibs}
\end{center}

Two or more stones can fusion together and form a group who share the liberties. Here we can see three group, one black and two white. The black one has five liberties by example. 

\begin{center}
\includegraphics[scale=0.5]{gogroupe}
\end{center}

An important concept is the capture, the board being representative(as a lot of ancient board games, like chess e.g) of a real situation of war. If a group is surrounded by the opponent, understand has no more liberties, it's dead. All the stones of the group are taken as prisoners, leaving the intersections where they were free. 

To win, you must have more point than your opponent. You gain one points by prisoner and one points by territory. A territory is a free intersection surrounded at least partially by your stones and potentially by the sides of the board. 

\begin{center}
\includegraphics[scale=0.4]{scorego}
\end{center}
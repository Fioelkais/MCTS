\section{Go and artificial intelligence}

Go is a really ancient board game. It is more than 2500 years old. Born in ancient china, it has relatively simple rules but a deep game complexity.  Actually they are more than 40 millions of players in the world. As chess, it's a game of strategy, which take its roots in military play. 
\\

Artificial intelligence on the other way, is the study and design of intelligent agent. There are several approaches behind the word intelligent. Systems can try to act or think,  like a human or rationally. Each of the four cases has its own approach and way of thinking. 
\\

Here we will present the development of an artificial intelligence that will finally be able to play Go. It will be more related to the part of A.I that is trying to act rationally. A.I and Go were always an interesting match, because until recently ( march 2016) the human players were better than the machine. Go was one of the last board games where it was the case. It was due to his complexity ( we estimate the number of possible games in chess at $10^200$  compared to $10^700$ in Go).  The usual algorithms couldn't take the challenge. It forced programmers to think deeper and about new techniques. That's where Monte Carlo tree search came. It helped a lot and allowed big progress in this field. The challenge was close this semester, thanks to Alpha Go, the bot of Google who defeated the world champion of Go. It used deep learning (it learned from a lot of humans games to know the best plays) to crack the problem. 
\\

In this chapter thesis, we will try to focus on the complexity of the modelling of Go in A.I. We'll show the state of the art and explain the main concept, the Monte Carlo tree search. Then we will present how we can progressively reached the lower bound of the complexity for representing a game of Go, software speaking. 

\section{Rules of Go}

The rules of Go are extremely simple, and their simplicity allows a real complexity and depth of play. The game is played on a so called go-ban, which is a board composed of 9*9,13*13 or 19*19 intersections. There are two players, one who play the white stones, the other the black stones. Each player puts a stone on an intersection on the go-ban alternatively.They are two important concepts to understand, the liberties and the group. 
\\

A stone has as many liberties as there are free intersections next to it. In figure 1, the black stone has initially four liberties, and it progressively decreases to one : 

\begin{figure}
\begin{center}
\includegraphics[scale=0.5]{Golibs}
\includegraphics[scale=0.5]{gogroupe}
\caption{Liberties of a stone and of a group}
\end{center}

\end{figure}
Two or more stones can fusion together to form a group in which they share the liberties. Here we can see three groups, one black and two white. The black one has five liberties. 


An important concept is the capture, the board being representative(as a lot of ancient board games, like chess e.g) of a real situation of war. If a group is surrounded by the opponent, it has no more liberty, it is dead. All the stones of the group are taken as prisoners, leaving free the intersections where they used to be. 

In order to win, a player must have more points than the opponent. A player gains one point by prisoner and one point by territory. A territory is a free intersection surrounded at least partially by your stones and potentially by the sides of the board. 
\\

In the example in \ref{score}, the white player win with 10 points ( 8 from territories and 2 from prisoners) against 	6 points for black( 5 from territories and 1 prisoner ). 
\begin{figure}
\begin{center}
\includegraphics[scale=0.5]{scorego.PNG}
\end{center}
\caption{Scoring}
\label{score}
\end{figure}